\chapter{Einleitung}

  \section{Vorstellung innoscale AG}
  Die \textit{innoscale AG} ist ein im Jahre 2013 gegründetes Unternehmen aus
  Berlin.  Spezialisiert auf das Stammdatenmanagement ist es sowohl beratend
  als auch mit einem eigenen Produkt zur Verbesserung der Datenqualität namens
  \textit{DataRocket}\footnote{Siehe Abschnitt: Vorstellug
  DataRocket~\ref{sec:Vorstellung-DataRocket}} tätig. Im Rahmen der
  Datenqualitätsverbesserung ist es nötig die Qualität von zuvor strukturierten
  Daten anhand von Kennzahlen auszuwerten und zu visualieren.

\section{Status quo und Problemstellung}
\begin{itemize}
  \item Die Daten sind vorhanden
  \item Auswertung bisher nur mit dem nötigen EDV-Wissen möglich
  \item Für technischen Laien sind die Daten unbrauchbar
  \item Daten enthalten wirtschaftlich relevante Fakten
  \item Daten sind komplex
\end{itemize}

\section{Ziel}
\begin{itemize}
  \item Spezifikation einer Datenanalysepraktik zugeschnitten auf das
    Unternehmen \textit{innoscale AG}
  \begin{itemize}
    \item Leicht verständlich
    \item Modellierung von Datenstrukturen, Kennzahlen und die Kombination der
      beiden als Fakten/Tabelle
    \item Datenstruktur generiert aus einem SQL-Schema
    \item Kennzahlen: Modellierung aller gängigen arithmentischen Operationen
    \item Fakten/Tabelle: mit Zeilen und Spalten aus Datenstruktur und Kennzahl 
  \end{itemize}
\item DataRocket-Komponente zur Konzeption von Datenauswertungen
  \begin{itemize}
    \item Spezifikation von Datenstrukturen, Kennzahlen und Tabellen
    \item Leicht verständlich und grafisch ansprechend
    \item Benutzbar für technischen Laien
  \end{itemize}
\end{itemize}
