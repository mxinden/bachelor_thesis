\chapter{notes}
\paragraph{}
\begin{itemize}
  \item Beispieldatenbank AdventureWorks-Datenbank von Microsoft
  \item ADAPT (Modellierung)
\end{itemize}

\paragraph{Aufteilung}
Am meisten soll \textit{inoscale Sprache und Methode} und \textit{DataFurnace} einnehmen. \textit{Existierende Methoden und Sprachen} soll nur einen kleinen Teil (max 3 Seiten) einnehmen.

\paragraph{Abgabe}
Am 30. September muss ich drei ausgedruckte Exemplare der BA bei Stefan im Büro
abgeben. Zusätzlich soll ich es ihm per E-Mail als PDF senden. Falls
zusätzliche digitale Quellen vorhanden sind, soll ich eine CD machen, nur dann,
wenn ich schon eine erstelle, soll ich noch den Source-Code als ZIP mit auf die
CD packen. Eine E-Mail an Stefan schicken, falls ich weiß, wann genau ich
abgeben möchte.

Die Arbeit soll 40 +- 10\% Seiten haben, ohne Inhaltsverzeichnis, Literaturverzeichnis, \ldots

\todo[inline]{Müssen die digitalen Fassungen unterschrieben werden? Nein}
\todo[inline]{Soll ich Screenshots vom Tool in den Anhang einfügen? Ja}
\todo[inline]{\textit{innoscale AG} kursiv schreiben? Ja}
\todo[inline]{Formatierung on \\paragraph in Ordnung? Folgewort ist in selber Zeile. Ja, wenn konsistent, und nicht nur ein Blatt im Überschriftenbaum}
\todo[inline]{Siehe klein schreiben in einer Klammer? Ja}
\todo[inline]{Sollte hinter Abbildung 1.~ein Punkt? Formatvorlage Word checken!}
\todo[inline]{Keine Zitate in Abbildungsverzeichnis}
\todo[inline]{Darf das Einleitungskapitel `Einleitung' heißen? Nein}
\todo[inline]{Muss ich einen Anhang haben? Nein}
\todo[inline]{Die Verwendung von ff ist nicht gestattet}
\todo[inline]{Man, soll, Füllwörter raus!}
\todo[inline]{Wo kommen die ganzen Anforderungen her? Workshop. Treffen mit Productowner, Ableitung aus Kundenterminen}
\todo[inline]{Vorgehen statt Prozess}
\todo[inline]{Limitation Feedback leider nur zu der Software. Für Feedback zu der Sprache müssen weitere Evaluationsläufe durchgeführt werden. Mein Schwerpunkt lag auf dem Tool}
\todo[inline]{Zeitpunkte aufschreiben, wann ich mit dem Personen gesprochen habe}

Taxonomie / Designsicience 
Mit Stefan absprechen, ob ich für die Entwicklung der Sprache eine konkrete
Methode brauche

\textbf{KEIN QUELLCODE}
