\chapter{Bachelorarbeit - Proposal}


\section{Ziele}
\begin{itemize}
  \item Konzeption einer \textbf{Datenstruktur-Metasprache} angelehnt an
    \textit{H2 for reporting}, welche sich nach den Ansprüchen 
    \begin{itemize}
      \item Abbildung komplexer hierarchischer Datenstrukturen
      \item Beschreibung ihrer modularen Beziehungen sowie
      \item Spezifikation von Kennzahlen und deren Berechnung auf Basis
        verschiedener Ebenen der hierarchischen Datenstruktur
    \end{itemize}
    von innoscale richtet.
  \item Entwicklung einer \textbf{Web-Applikation} zur Anwendung der
    Datenstruktur-Metasprache welche
    \begin{itemize}
      \item die Spezifikation der Datenstruktur eines \acrshort{DWH} und
      \item die Spezifikation von Berichten aus einer spezifizierten
        \acrshort{DWH} Datenstruktur
    \end{itemize}
    ermöglicht. Diese Applikation soll für den Benutzer grafisch ansprechend
    und leicht zu bedienen sein.
\end{itemize}


\section{Vorgehen}
\begin{enumerate}
  \item Vergleich verschiedener Sprachdialekte (\textit{H2 for reporting} etc.)
  \item Sichtung weiterer Literaturquellen zum Thema Metadatenmanagement
  \item Erkunden des H2 Toolsets im Zusammenhang mit \textit{H2 for
    reporting}
  \item Konzeption der innoscale spezifischen Datenstruktur-Metasprache in
    enger Absprache mit innoscale
  \item Konzeption der Web-Applikation
  \item Entwicklung der Web-Applikation
  \item Ausblick: Integration der entwickelten Lösung in DataRocket
    (konzeptionell)
\end{enumerate}

